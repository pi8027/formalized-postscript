\documentclass[cjk, 14pt, dvipdfm]{beamer}

\usepackage{color}
\usepackage{amsmath}
\usepackage{listings}
\usepackage{stmaryrd}

\usetheme{Copenhagen}
\usecolortheme{seahorse}
\useinnertheme{rounded}
\useoutertheme{shadow}

\setbeamercovered{transparent}

\renewcommand\kanjifamilydefault{\gtdefault}
\renewcommand\familydefault{\sfdefault}

\definecolor{code-background}{gray}{0.8}

\lstdefinestyle{plain}{
  basicstyle=\small\tt,
  keywordstyle=,
  identifierstyle=,
  commentstyle=,
  stringstyle=,
  emphstyle=,
  backgroundcolor=\color{code-background},
  language=,
  frame=trbl,
  rulecolor=\color{white},
  numbers=none,
  numberstyle=,
  xleftmargin=0.4zw,
  xrightmargin=0.4zw,
  basewidth={0.48em, 0.43em},
  lineskip=-0.2ex
}

\lstdefinestyle{postscript}{
  style=plain,
  keywordstyle=,
  identifierstyle=,
  commentstyle=,
  stringstyle=,
  emphstyle=,
  language=PostScript,
}

\newlength{\lengthwithlength}
\newcommand{\bnfvert}
    {\settowidth{\lengthwithlength}{::=}\mathrel{\makebox[\lengthwithlength][c]{$|$}}}
\newcommand{\bnfcce}{\mathrel{::=}}

\title{\large{計算の状態を見ながらプログラミング\\できるプログラム導出システム}}
\author{坂口和彦}
\institute{情報学群 情報科学類 B2}
\date{2013/01/21}

\begin{document}

\begin{frame}[plain]

 \maketitle

\end{frame}

\begin{frame}[plain]

 \begin{itemize}
  \item 目標
  \begin{itemize}
    \item 通常であればプログラムを書いているときに見えない「プログラムの実行中の状態」が見えるプログラミング環境を開発する
  \end{itemize}
  \item 方法
  \begin{itemize}
    \item SSReflect-Coqという定理証明器を使って実装
  \end{itemize}
  \item 結果
  \begin{itemize}
    \item 状態が見えるだけでなく証明が書ける
    \item 自明な計算を自動生成する仕組みを開発した
    \item ライブラリをそれなりに充実させた
    \item この仕組みの上で書いたプログラムをPostScriptに変換できる
    \item \footnotesize\texttt{http://github.com/pi8027/formalized-postscript}
  \end{itemize}
 \end{itemize}

\end{frame}

\end{document}
