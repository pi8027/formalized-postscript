\documentclass[cjk, 14pt, dvipdfm]{beamer}

\usepackage{color}
\usepackage{amsmath}
\usepackage{listings}
\usepackage{stmaryrd}

\usetheme{Copenhagen}
\usecolortheme{seahorse}
\useinnertheme{rounded}
\useoutertheme{shadow}

\setbeamercovered{transparent}

\renewcommand\kanjifamilydefault{\gtdefault}
\renewcommand\familydefault{\sfdefault}

\definecolor{code-background}{gray}{0.8}

\lstdefinestyle{plain}{
  basicstyle=\small\tt,
  keywordstyle=,
  identifierstyle=,
  commentstyle=,
  stringstyle=,
  emphstyle=,
  backgroundcolor=\color{code-background},
  language=,
  frame=trbl,
  rulecolor=\color{white},
  numbers=none,
  numberstyle=,
  xleftmargin=0.4zw,
  xrightmargin=0.4zw,
  basewidth={0.48em, 0.43em},
  lineskip=-0.2ex
}

\lstdefinestyle{postscript}{
  style=plain,
  keywordstyle=,
  identifierstyle=,
  commentstyle=,
  stringstyle=,
  emphstyle=,
  language=PostScript,
}

\newlength{\lengthwithlength}
\newcommand{\bnfvert}
    {\settowidth{\lengthwithlength}{::=}\mathrel{\makebox[\lengthwithlength][c]{$|$}}}
\newcommand{\bnfcce}{\mathrel{::=}}

\title{定理証明器Coqによる\\PostScriptプログラミング}
\author{坂口和彦 (@pi8027)}
\institute{情報科学類 B2}
\date{2012/12/21}

\begin{document}

\begin{frame}[plain]

 \maketitle

\end{frame}

\begin{frame}{概要}

 \begin{itemize}
  \item 定理証明器SSReflect-CoqでPostScriptのサブセットを定義し、その上で証明付きのプログラムを記述
	する
  \item ソースコード、スライド: \small\texttt{http://github.com/pi8027/formalized-postscript}
 \end{itemize}

\end{frame}

\begin{frame}{SSReflect-Coq}

 \begin{itemize}
  \item Coqはコンピュータの上でプログラムや形式的な証明を記述するためのソフトウェア
  \item SSReflectはCoqをより便利かつ強力な道具にするための拡張とライブラリ
  \item ソフトウェアやその周辺技術の検証の道具としてCoqを使う例は多い
 \end{itemize}

\end{frame}

\begin{frame}{PostScript}

 \begin{itemize}
  \item ベクタ画像を記述するための形式
  \item スタックの指向プログラミング言語
  \item プリンタとかGhostScriptで実行できる
 \end{itemize}

\end{frame}

\begin{frame}[fragile]{PostScriptプログラムの例}

\begin{lstlisting}[style=postscript]
> 1 2 add =
3
> 3 dup mul dup mul =
81
> 10 20 exch sub =
10
> 10 [ exch /add cvx ] cvx dup ==
{10 add}
> 10 1 index exec = 20 1 index exch =
20
30
\end{lstlisting}

\end{frame}

\begin{frame}{目的}

 \begin{itemize}
  \item 証明を付けられる形でPostScriptに似た意味を持つ言語を実装したい
  \item 実行中のプログラムの(十分に一般化された)状態を見ながらプログラムを書きたい
  \item 実装の自明なプログラムはより直感的な仕様から自動生成したい
  \pause
  \item これらの目的は全て達成できた
 \end{itemize}

\end{frame}

\begin{frame}{言語の定義 - 1}

 \begin{align*}
  i &\bnfcce  \mathit{pop}                & \text{値を捨てる} \\
    &\bnfvert \mathit{copy}               & \text{値を複製する} \\
    &\bnfvert \mathit{swap}               & \text{2つの値を入れ替える} \\
    &\bnfvert \mathit{cons}               & \text{2つの値を\textit{pair}で組にする} \\
    &\bnfvert \mathit{quote}              & \text{値を\textit{push}で包む} \\
    &\bnfvert \mathit{exec}               & \text{値を実行する} \\
    &\bnfvert \mathit{push} \, i          & \text{$i$をスタックに積む} \\
    &\bnfvert \mathit{pair} \, i_1 \, i_2 & \text{$i_1$と$i_2$を順番に実行する}
 \end{align*}

 \begin{itemize}
  \item スタックは命令のリスト
  \item 2本のスタックの組を状態と呼ぶ
 \end{itemize}

\end{frame}

\begin{frame}{言語の定義 - 2}

 \begin{align*}
  i_1 :: vs , \mathit{pop} :: cs           & \Mapsto vs , cs \\
  i_1 :: vs , \mathit{copy} :: cs          & \Mapsto i_1 :: i_1 :: vs , cs \\
  i_2 :: i_1 :: vs , \mathit{swap} :: cs   & \Mapsto i_1 :: i_2 :: vs, cs \\
  i_2 :: i_1 :: vs , \mathit{cons} :: cs   & \Mapsto (\mathit{pair} \, i_1 \, i_2) :: vs , cs \\
  i_1 :: vs , \mathit{quote} :: cs         & \Mapsto (\mathit{push} \, i_1) :: vs , cs \\
  i_1 :: vs , \mathit{exec} :: cs          & \Mapsto vs , i_1 :: cs \\
  vs , (\mathit{push} \, i_1) :: cs        & \Mapsto i_1 :: vs , cs \\
  vs , (\mathit{pair} \, i_1 \, i_2) :: cs & \Mapsto vs , i_1 :: i_2 :: cs
 \end{align*}

\end{frame}

\begin{frame}[fragile]{実装}

\begin{lstlisting}[style=plain, basicstyle=\tt]
$ git log --oneline --all | wc -l
172
$ wc *.v
   51   225  1253 Common.v
  174   684  4703 PsBool.v
  412  1733 11944 PsCore.v
   69   261  1588 PsExample.v
  384  1810 12253 PsNat.v
  393  2212 12865 PsTemplate.v
 1483  6925 44606 total
\end{lstlisting}

\end{frame}

\begin{frame}{プログラミングをより手軽に}

 \begin{itemize}
  \item Coq上で証明したプログラムをPostScriptに変換可能
  \item PostScriptのプログラムにCoqの式を埋め込む
  \item ライブラリも充実!
	\begin{itemize}
	 \item と言いたいところだが、まだブール型と自然数に関するものしかない
	\end{itemize}
 \end{itemize}

\end{frame}

\begin{frame}{これまでとこれから}

 \begin{itemize}
  \item 2012-06-14: 最初のコミット
  \item 2012-08-11: コミックマーケット82
  \item 2012-09-02: Proof Summit 2012
  \item 2012-09-16: 第45回情報科学若手の会
  \item 2012-11-22: TPP2012
  \item \textcolor{red}{2012-12-21: 情報特別演習 最終発表会}
  \item \dots
	\begin{itemize}
	 \item ライブラリを充実させる
	 \item 他の良く知られた計算体系を模倣できることを証明する
	 \item より良い形で残したい
	\end{itemize}
 \end{itemize}

\end{frame}

\end{document}
